\documentclass[doc]{apa6}
%%%%%%%%%%%%%%%%%%%%%%%
%Anfang Verweise
\usepackage[ngerman]{varioref}                                                   %elegantere Verweise
\usepackage[hidelinks]{hyperref}                                                  %Anklickbare Links/Verweise
\usepackage[ngerman]{cleveref}                                                   %Kontextbasierte Verweise
\usepackage{nameref}                                                    %Verweist auf die Überschrift mit Namen
%   \usepackage{showkeys}
%Ende Verweise
%%%%%%%%%%%%%%%%%%%%%%%

%%%%%%%%%%%%%%%%%%%%%%%
%Anfang Markierungen
\usepackage{xcolor}                                                     %Fügt Farbe hinzu
\usepackage[disable]{todonotes}                                                  %Todoliste [disable]
\usepackage{blindtext}                                                  %Blindtexte
%Ende Markierungen
%%%%%%%%%%%%%%%%%%%%%%%

%%%%%%%%%%%%%%%%%%%%%%%%%
%Begin Formale Gestaltung
\usepackage{indentfirst}                                                %Einzug bei jedem Absatz
\usepackage[onehalfspacing]{setspace}                                    %Doppelter Zeilenabstand
\usepackage{microtype}                                                  %Exakter Blocksatz
\usepackage[left=1cm,right=1.5cm,top=2cm,bottom=1.5cm,includeheadfoot]{geometry}
%Ende Formale Gestaltung
%%%%%%%%%%%%%%%%%%%%%%%%

%%%%%%%%%%%%%%%%%%%%%%%%%%%%%%%%%%%%%%%%%
%Anfang Sprache, Compiler, Zitation
\usepackage[T1]{fontenc}                                                %T1 Support
\usepackage[utf8]{inputenc}                                             %Unicode Support
\usepackage[babel, german=guillemets]{csquotes}                         %Einfache Anführungszeichen
    %Biblatex Zitationen
    \usepackage[style=apa, hyperref=true]{biblatex}                     %Zitierumgebung
    \DeclareLanguageMapping{ngerman}{ngerman-apa}                       %Zitationsstil
    \addbibresource{lit.bib}                                            %Literaturverzeichnis
    
\usepackage[ngerman]{babel}                                             %DeutscheVoreingestellt
%Ende Sprache, Interpretation, Zitation
%%%%%%%%%%%%%%%%%%%%%%%%%%%%%%%%%%%%%%%



\hypersetup{
  pdftitle    = {Sexualstraftäter: Der krankhafte Sadist},
  pdfsubject  = {Forensische Psychatrie},
  pdfauthor   = {S. Kotowski},
}

\begin{document}
\large
    \title{ Sexualstraftäter: Der krankhafte Sadist}
    \author{Silas Kotowski}
    \affiliation{Friedrich-Alexander Universität Erlangen-Nürnberg}
    \maketitle
    
    
    \section{Einleitung}
        
        Auch wenn es viele Elemente gibt die ein Risiko für frühzeitige Sterblichkeit darstellen, wurden soziale Faktoren in der wissenschaftlichen Literatur deutlich weniger berücksichtigt als andere, obwohl diese nachweislich gleichen oder größeren Einfluss auf die Mortalität haben.
        Sozialer Kontakt hat sich nicht nur für das  emotionale und psychische Wohlbefinden als förderlich erwiesen, sondern auch für das physische.  Insgesamt sind die schützenden Effekte von sozialen Beziehungen hinsichtlich früher Sterblichkeit bekannt.
        
        Im Rahmen dieses Artikels soll unser Verständnis zu den Gefahren von sozialen Defiziten untersucht werden. In der bisherigen Forschung wurde davon ausgegangen, dass die positiven Effekte von sozialen Kontakten umgekehrt den negativen Effekten von sozialen Defiziten entsprechen oder, ob diese negativen Effekte überwiegen. Außerdem gibt es bisher keine Metaanalyse, die sich - unter Beobachtung der Mortalität - auf soziale Isolation und Einsamkeit fokussiert hat.

        Um die mögliche Gefahr der frühzeitigen Mortalität minimieren zu können müssen erst die Zusammenhänge verstanden werden. Damit soll diese Studie ein erster Schritt dafür sein den Fokus auf die Wichtigkeit dieser Gefahr zu lenken und es so später zu ermöglichen Risikogruppen frühzeitig zu erkennen und ihnen helfen zu können.

    \section{Theorie}
            In der Metaanalyse wurden Werten für soziale Isolation und Einsamkeit und die Tatsache, ob der Proband alleine lebte nach einem Zusammenhang zu frühzeitiger Mortalität untersucht. 
            
            Die Unterscheidung zwischen sozialer Isolation und Einsamkeit wurde getroffen, da sich in vorhergehenden Studien gezeigt hatte, dass diese beiden oft nicht signifikant miteinander korrelieren, was die Autoren zu dem Schluss führt, dass es sich hierbei um unabhängige Konstrukte handeln könne, welche getrennt voneinander untersucht werden müssen.
            
            

        
    \section{Methode}
        Die Autoren suchten zuerst nach Studien, welche Schlüsselworte wie \glqq Mortalität\grqq, \glqq tot\grqq, \glqq am leben geblieben\grqq und ein Synonym von sozialer Isolation, Einsamkeit oder alleine leben enthielten. In der Metaanalyse sind die Studien daraus ausgeschlossen, welche Mortalität als Ergebnis von Suizid oder Unfällen hatten oder deren Ergebnis sich nicht auf Mortalität beschränkte, sondern zum Beispiel noch Morbidität erfasste. Berücksichtigt wurden Studien, die in englischer Sprache verfasst waren und quantitative Daten bezüglich der Mortalität als Ergebnis eines objektiven oder subjektiven Markers für soziale Isolation untersuchte. Im Sinne dieser Studie bedeutet das Mortalitätsrisiko also eine Schätzung des Ausmaßes zu welchem soziale Isolationsmarker die Wahrscheinlichkeit des Todes bei einem follow-up vorhersagen kann.
        
        Zur Abstraktion der Daten haben je zwei Teams aus je zwei Ratern jeden Artikel zweimal codiert und a) die Anzahl der Studienteilnehmer und deren Zusammensetzung nach Alter, Gender, Gesundheitsstatus, Vorerkrankungen und Todesursache, b) die Länge des follow-ups, c) das Untersuchungsdesign, d) die Art der untersuchten sozialen Isolation (tatsächliche oder empfundene), e) die Anzahl und Art der einbezogenen Drittvariablen in die statistische Auswertung aus jeder Studie extrahiert. Ebenso wurden f) die Ausschlüsse von Probanden durchgeführt, die schwer krank waren oder sehr schnell nach Studienbeginn gestorben waren. 
        
        Aus jeder Studie wurde weiterhin die Odds Ratio aus jeder Studie extrahiert, wobei bei mehreren Effektgrößen zu einem Zeitpunkt nach Standardfehler gemittelt wurden und bei mehreren Messzeitpunkten, die Effektgröße bei der letzen Follow-Up gewählt wurde. Für den Fall, dass mehrere Effektgrößen für unterschiedliche Ausmaße sozialer Isolation auftraten, wurde die ausgewählt, welche den höchsten Kontrast hatte (hohe Isolation vs niedrige Isolation).
        
        Die Interraterübereinstimmung war sowohl für die  kategorialen Daten (Cohens Kappa durchschnittlich .73), als auch für kontinuierliche Daten (Cohens Kappa durchschnittlich .95) ausreichend hoch.
        
    \section{Ergebnis}
        
        Die Effektgrößen in den 70 Studien wurde von den Forschenden auf verschieden Arten berechnet, wobei einige unbereinigte Werte angaben und andere eine Vielzahl an Drittvariablen einbezogen. Die Odd Ratios lagen zwischen 0.64 und 3.85 mit einer außerordentlich hohen Heterogenität zwischen den Studien, weshalb die Autoren dieser Studie die Analyse nach der Anzahl der genutzten Drittvariablen aufteilten. So ergaben 31 die Untersuchungen, die keine Drittvariablen einbezogen über alle drei Prädiktoren (soz. Isolation, Einsamkeit und Alleine leben) eine höhere Mortalität bei Probanden mit höherer sozialer Isolation (\textit{OR} = 1.53, 95\%CI: 1.38, 1.70).
        
        Ähnliche Werte fanden sich bei teilweise bereinigten Daten aus 21 Studien mit einer Odd Ratio von 1.51 ebenso (95\%CI: 1.27, 1.79). 
        
        Auch die bereinigten Daten aus insgesamt 52 Studien zeigten eine höhere Mortalität bei objektiv oder subjektiv sozial stärker isolierten Probanden (\textit{OR} = 1.30, 95\% CI: 1.16, 1.46). 
        
        Insgesamt zeigte jedes Maß (soziale Isolation, Einsamkeit und alleine leben) für jeden Datentyp (unbereingit, teilweise bereinigt und ganz bereinigt) eine OR zwischen 1.26 und 1.83. Diese drei Maße unterschieden sich in ihrer Aussagekraft in keiner der drei Datentypen, was die Autoren dahingehend interpretierten, dass es keinen allgemeinen Unterschied zwischen der Aussagekraft der beiden objektiven und dem einen subjektiven Prädiktor gebe.
        
        Die einzelnen Datenarten unterschieden sich hingegen. So zeigten die unbereinigten Daten signifikant größere Effektstärken als die ganz bereinigten Daten (\textit{p}~<~.001). 
        
        Die Untersuchung möglicher Moderatoren wurde nur mit den ganz bereinigten Daten durchgeführt um mögliche Konfundierungen bezüglich der Mortalität zu vermeiden. Dabei stellte sich heraus, dass die acht Studien, die für Drittvariablen kontrollierten, die direkt mit sozialer Unterstützung zusammenhingen (Ehestatus, Soziales Netzwerk und Einsamkeit) geringere Effektgrößen verzeichnet wurden, als für die anderen 33 Studien, die keine derartigen Variablen berücksichtigen. Die anderen kategorialen Variablen zeigten keinen bemerkenswerten Effekt auf die Effektgröße.
        Bei den kontinuierlichen Variablen zeigten keinen größeren Zusammenhang als .20 zwischen Effektgröße und Beginn der Datensammlung, Followup-Länge und Anteil der weiblichen Probanden.
        Allerdings führte eine Kontrolle von sieben oder mehr Drittvariablen zu einem homogeneren Ergebnis in den Studien, ohne jedoch die generelle Effektgröße zu beeinträchtigen.
        
        Bemerkenswert war hingegen noch, dass die Effektstärke in Studien mit einem Durchschnittsalter von unter 65 Jahren (\textit{OR}~=~1.57 bzw. 1.92)  sowohl bei unbereinigten, als auch bei bereinigten Daten, deutlich höher lag als in Studien mit Stichproben mit einem Durchschnittsalter von über 75 Jahren (\textit{OR}~=~ 1.14 bzw. 1.28).
        
        Ebenfalls wurde der Publication Bias kontrolliert und es stellte sich heraus, dass 1,268 unpublizierte Studien keinen Effekt erzielt hätten haben müssen, damit die Aussagekraft dieser Metaanalyse gefährdet würde.
        
    \section{Diskussion}
        Die Studie zeigt, dass die drei Marker für soziale Isolation (objektive soziale Isolation, subjektive Einsamkeit und alleine Leben) in den 70 unabhäniggen Studien mit insgesamt 3,4 Millionen Probanden innerhalb von 7 Jahren einen signifikanten Anstieg der Mortalität begleiten. Im Schnitt lag das Risiko für den Tod bei isolierten Probanden um 29\% höher als bei den anderen Probanden. Des Weiteren zeigten die drei Maße für soziale Isolation keinen nennenswerten Unterschied in ihrer Aussagekraft. 
        
        Beachtenswert ist, dass die Effektstärke für die jüngeren Probanden höher ist als für ältere, wofür die Autoren mehrere Erklärungen anführen. 1) Die Probanden, die 75 Jahre alt geworden sind, sind resilienter gegen soziale Isolation als die jüngeren 2) mit der Rente ist es für Menschen normal, dass sich deren soziales Netzwerk verkleinert, sodass dies für sie kein Problem ist 3) einsame Menschen neigen weniger zu gesundheitsförderlichem Verhalten und sterben deshalb früher 4) es könnte sich bei diesem Effekt um eine Konfundierung mit dem Ehestatus handeln.
        
        Kritisch muss bei dieser Metaanalyse beäugt werden, dass das durchschnittliche Alter der Probanden meist weit über 50 Jahren lag, was wohl aber zum einen an der Ausgangslage der vorhanden Studien lag, als auch an dem Umstand, dass Mortalitätseffekte in jüngeren Jahren deutlich schwerer zu detektieren sind.
        
    
    \section{Reflexion der Seminarsitzung}
        Hinsichtlich der Frage eines bedingsungslosen Grundeinkommens, bei dem mehr Arbeit keine zusätzliche Entlohnung bringt mag es zuerst schwierig sein diese Studie einzuordnen. Doch wenn man überlegt, dass die Arbeit etwa 10\% der Lebenszeit einer 80jährigen Person ausmacht, stellt man fest, dass dieser Ort auch ein sozialer Treffpunkt ist. Der Austausch mit Kollegen oder auch Freundschaften die man auf der Arbeit knüpft, würden wegfallen, wenn die Arbeit weniger attraktiv ist.
        
        Natürlich kann man argumentieren, dass der Mensch dies merkt und bald feststellt, dass er besser arbeiten sollte, und sei es nur zur Pflege sozialer Kontakte. Doch wenn wir ehrlich sind, sollten Menschen sich auch gesünder ernähren, mehr Sport machen und weniger prokrastinieren. All dies ist uns jetzt auch bewusst, mit geringem Ergebnis.
        
        Haben wir nun die Arbeit als sozialen Treffpunkt etabliert müssen wir uns auch darüber bewusst werden, wie wir überhaupt ein soziales Netzwerk aufbauen. Sicherlich gibt es extravertierte Personen, die leicht neue Kontakte schließen können und schnell einen neuen Bekannten- und Freundeskreis aufgebaut haben, selbst wenn sie umgezogen sind. Doch die meisten Kontakte beginnen meiner Ansicht nach, in Schule, Universität und Beruf mit den Menschen, mit denen ich täglich, mehr oder weniger freiwillig zu tun habe. Hier drängt sich mir der Vergleich zum Schulsport auf, der auch für alle Schüler verpflichtend ist. Man könnte die Arbeit demnach als sozialen Schulsport ansehen, in dem man, zum eigenen besten, gezwungen ist, mit anderen zu interagieren und zu kommunizieren.
        
        Aus diesen Gründen bin ich gegen ein bedingungsloses Grundeinkommen ohne Zuverdienstmöglichkeit.
        
        Um diese Thesen zu stützen, würde sich auch weitere Forschung anbieten, etwa, ob arbeitslose Menschen ein kleineres soziales Netzwerk haben als Arbeitstätige. Ebenfalls könnte untersucht werden, wo die meisten Kontakte einer Person ihren Ursprung haben.
        
        
        
\defbibheading{bibliography}{\section*{Literatur}}
    \printbibliography
\end{document}